\documentclass[11pt,a4paper]{article}

\usepackage[margin=1in, paperwidth=8.3in, paperheight=11.7in]{geometry}
\usepackage{amsmath,amsfonts,fancyhdr}
\usepackage[section]{DomH}
\headertitle{Stochastic Processes}

\begin{document}

\title{Stochastic Processes}
\author{Dom Hutchinson}
\date{\today}
\maketitle

\section{Stochastic Processes}

\begin{definition}{Stochastic Process}
  % sample space of stochastic process
  % discrete time stochastic process
  % continuous time
\end{definition}

\begin{definition}{Adapted Stochastic Process}

\end{definition}

\begin{definition}{State Space}

\end{definition}

\begin{definition}{Markov Property}

\end{definition}

\begin{definition}{Markov Process}

\end{definition}

\begin{definition}{Increasing \& Decreasing Sequences}

\end{definition}

\begin{definition}{Stopping Time}

\end{definition}

\begin{theorem}{Wald's Lemma}

\end{theorem}

\begin{definition}{Detailed Balanced Condition}

\end{definition}

\section{Random Walks}

\begin{definition}{Random Walk}

\end{definition}

\begin{remark}{As a Markov Chain}

\end{remark}

\begin{definition}{Simple Symmetric Random Walks}

\end{definition}

\begin{definition}{Absorbing Barriers}

\end{definition}

\begin{theorem}{One-Step Conditioning Argument}

\end{theorem}

\begin{definition}{Gambler's Ruin}

\end{definition}

\begin{theorem}{Probability of Ruin \w Absorbing Barriers}

\end{theorem}

\begin{definition}{Spatial Homogeneity}

\end{definition}

\begin{definition}{Null Recurrent}

\end{definition}

\begin{definition}{Positive Recurrent}

\end{definition}

\subsection{Brownian Motion}

\begin{definition}{Brownian Motion}
  % 1D
  % Standard
\end{definition}

\begin{definition}{Transition Density}

\end{definition}

\begin{theorem}{Reflection Principle}

\end{theorem}

\section{Martingales}

\begin{definition}{Discrete Time Martingales}

\end{definition}

\begin{definition}{Continuous Time Martingales}

\end{definition}

\begin{definition}{Super Martingale}

\end{definition}

\begin{definition}{Stopped Discrete Time Martingales}

\end{definition}

\begin{theorem}{Optional Stopping Theorem}

\end{theorem}

\begin{theorem}{Martingale Convergence Theorem}

\end{theorem}

\begin{remark}{Brownian Motion is a Martingale}

\end{remark}

\section{Markov Chains}

\begin{definition}{Markov Chain}
  % State
\end{definition}

\begin{definition}{Transition Matrix}
  % Properties
  % Automata
\end{definition}

\begin{definition}{Time-Homogenous Markov Chain}

\end{definition}

\begin{remark}{Lack of Memory Property}

\end{remark}

\begin{definition}{Irreducible Markov Chain}

\end{definition}

\begin{definition}{Generator}

\end{definition}

\begin{definition}{Forward Equation}

\end{definition}

\subsection{Transitions}

\begin{theorem}{$n$-Step Transitioning Probability}
  % generating function
\end{theorem}

\begin{theorem}{First Passage Probabilities}
  %Expected first passage
  % generating function
\end{theorem}

\begin{theorem}{Number of Visits}

\end{theorem}

\begin{definition}{Period}
  % Aperiodic
  % within a communication class
\end{definition}

\begin{theorem}{Chapman-Kolmogorov Equation}

\end{theorem}

\begin{remark}{Who Arrives First?}

\end{remark}

\begin{definition}{Jump Chain}

\end{definition}

\subsection{Properties of States}

\begin{definition}{Transient State}
  % continuous & discrete time
\end{definition}

\begin{definition}{Recurrent State}
% continuous & discrete time
\end{definition}

\begin{definition}{Closed State}

\end{definition}

\begin{definition}{Irreducible State}

\end{definition}

\begin{definition}{Absorbing State}

\end{definition}

\begin{definition}{Positive Recurrent State}

\end{definition}

\begin{definition}{Communication}
%intercommunication
\end{definition}

\begin{definition}{Communication Classes}
  % Closed communication class
  % Non-closed communication classes are transient
\end{definition}

\subsection{Stationary Distribution}

\begin{definition}{Stationary Distribution}

\end{definition}

\begin{definition}{Generator}

\end{definition}

\section{Gaussian Processes}

\begin{definition}{Gaussian Processes}

\end{definition}

\section{Poisson Process}

\begin{definition}{Poisson Process}
  % distribution
\end{definition}

\subsection{Birth-Death Processes}

\begin{definition}{Linear Birth Process}
  % Distribution of number of Births
  % Distribution of population size
\end{definition}

\begin{definition}{Linear Birth-Death Process}

\end{definition}

\begin{definition}{Generalist Birth-Death Process}

\end{definition}


\end{document}
