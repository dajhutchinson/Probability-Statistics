\documentclass[11pt,a4paper]{article}

\usepackage[margin=1in, paperwidth=8.3in, paperheight=11.7in]{geometry}
\usepackage{amsmath,amsfonts,fancyhdr}
\usepackage[section]{DomH}
\headertitle{Regression}

\begin{document}

\title{Regression}
\author{Dom Hutchinson}
\date{\today}
\maketitle

\section{General}

\begin{definition}{Hypothesis}
  % Simple Composite alternative null
\end{definition}

\begin{definition}{Hypothesis Testing}
  % one sided v two sided
\end{definition}

\begin{remark}{Types of Hypothesis Tests}

\end{remark}

\begin{proposition}{Hypothesis Testing Process}

\end{proposition}

\begin{definition}{Test Statistic}
  % equivalent test statistics
\end{definition}

\begin{definition}{Significance Level}

\end{definition}

\begin{definition}{$p$-Value}

\end{definition}

\begin{definition}{Critical Region}

\end{definition}

\begin{definition}{Critical Value}

\end{definition}

\begin{remark}{Using Confidence Intervals for Hypothesis Testing}
  % test inversion
\end{remark}

\subsection{General Approach}

\begin{proposition}{General Procedure for Hypothesis Testing}

\end{proposition}

\subsection{Neyman-Pearson Approach}

\begin{definition}{Neyman-Pearson Test Statistic}

\end{definition}

\begin{theorem}{Neyman-Pearson Lemma}

\end{theorem}

\begin{proposition}{Procedure for using Neyman-Pearson Approach}

\end{proposition}

\subsection{Generalised Likelihood Ratio Test}

\begin{definition}{Nested Parameter Space}

\end{definition}

\begin{definition}{Generalised Likelihood Ratio Test}

\end{definition}

\section{Assessing Hypothesis Tests}

\begin{definition}{Type 1 Error}

\end{definition}

\begin{definition}{Type 2 Error}

\end{definition}

\begin{definition}{Power of Hypothesis Test}

\end{definition}

\begin{definition}{Uniformly Most Powerful Test}

\end{definition}

\begin{definition}{Size of Test}

\end{definition}

\section{Comparing Population Means}

\begin{definition}{Types of Group Samples}

\end{definition}

\begin{definition}{Two Sample $t$-Test}

\end{definition}

\begin{definition}{Pooled Estimate Test}

\end{definition}

\section{Pearson's $\chi^2$ Test}

\begin{definition}{Pearson's $\chi^2$ Test Statistic}

\end{definition}

\section{Bayesian Hypothesis Testing}

\begin{definition}{Bayesian Hypothesis Testing}

\end{definition}

\section{Parameters of Linear Models}

\begin{theorem}{Distribution of $\hat{\pmb\beta}$}

\end{theorem}

\begin{remark}{Hypothesis Testing}

\end{remark}

\end{document}
